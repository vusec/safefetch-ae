%%%%%%%%%%%%%%%%%%%%%%%%%%%%%%%%%%%%%%%%%%%%%%%%%%%%%%%%%%%%%%%%%%%%%%%%%%%%%%%%
% Template for USENIX papers.
%
% History:
%
% - TEMPLATE for Usenix papers, specifically to meet requirements of
%   USENIX '05. originally a template for producing IEEE-format
%   articles using LaTeX. written by Matthew Ward, CS Department,
%   Worcester Polytechnic Institute. adapted by David Beazley for his
%   excellent SWIG paper in Proceedings, Tcl 96. turned into a
%   smartass generic template by De Clarke, with thanks to both the
%   above pioneers. Use at your own risk. Complaints to /dev/null.
%   Make it two column with no page numbering, default is 10 point.
%
% - Munged by Fred Douglis <douglis@research.att.com> 10/97 to
%   separate the .sty file from the LaTeX source template, so that
%   people can more easily include the .sty file into an existing
%   document. Also changed to more closely follow the style guidelines
%   as represented by the Word sample file.
%
% - Note that since 2010, USENIX does not require endnotes. If you
%   want foot of page notes, don't include the endnotes package in the
%   usepackage command, below.
% - This version uses the latex2e styles, not the very ancient 2.09
%   stuff.
%
% - Updated July 2018: Text block size changed from 6.5" to 7"
%
% - Updated Dec 2018 for ATC'19:
%
%   * Revised text to pass HotCRP's auto-formatting check, with
%     hotcrp.settings.submission_form.body_font_size=10pt, and
%     hotcrp.settings.submission_form.line_height=12pt
%
%   * Switched from \endnote-s to \footnote-s to match Usenix's policy.
%
%   * \section* => \begin{abstract} ... \end{abstract}
%
%   * Make template self-contained in terms of bibtex entires, to allow
%     this file to be compiled. (And changing refs style to 'plain'.)
%
%   * Make template self-contained in terms of figures, to
%     allow this file to be compiled. 
%
%   * Added packages for hyperref, embedding fonts, and improving
%     appearance.
%   
%   * Removed outdated text.
%
%%%%%%%%%%%%%%%%%%%%%%%%%%%%%%%%%%%%%%%%%%%%%%%%%%%%%%%%%%%%%%%%%%%%%%%%%%%%%%%%

\documentclass[letterpaper,twocolumn,10pt]{article}
\PassOptionsToPackage{hyphens}{url}
\usepackage{hyperref}
\usepackage{usenix2019_v3}


%\hypersetup{
%    pdfborder = {0 0 0}
%}

% to be able to draw some self-contained figs
\usepackage{tikz}
\usepackage{amsmath}

\usepackage{todonotes}
\usepackage{verbatim}

% inlined bib file
\usepackage{filecontents}

% From previous paper template

\hyphenation{op-tical net-works semi-conduc-tor}
\usepackage{amsmath,amssymb,amsfonts}
\usepackage{textcomp}
\usepackage{xcolor}
\usepackage{xspace}
\usepackage{enumerate}
\let\labelindent\relax
\usepackage{enumitem}
\usepackage{float}
\usepackage{subfloat}
\usepackage{subfigure}


\ifdefined\directlua
  \usepackage{fontspec}
\else
  \usepackage[T1]{fontenc}
  \usepackage[nomath]{lmodern}
\fi

% Paper defined names
\newcommand{\techdefault}[0]{\texttt{SafeFetch-default}\xspace}
\newcommand{\techwhitelist}[0]{\texttt{SafeFetch-whitelist}\xspace}
\newcommand{\tech}[0]{\texttt{SafeFetch}\xspace}
\newcommand{\techlong}[0]{SafeFetch\xspace}
\newcommand{\sanitizercomponent}[0]{\textbf{\texttt{Cache Frontend}}\xspace}
\newcommand{\cachemanager}[0]{\texttt{\textbf{Cache Backend}}\xspace}
\newcommand{\xxx}[1]{\textcolor{orange}{\textbf{#1}}} % version 1

% End of defined names

\usepackage{textcomp}
\usepackage{listings}
\usepackage{multirow}
\usepackage[normalem]{ulem}

\lstset{
    basicstyle=\footnotesize\ttfamily,
    escapeinside={/*!}{!*/}
}
\newcounter{lstannotation}
\setcounter{lstannotation}{0}
\renewcommand{\thelstannotation}{\ding{\number\numexpr181+\arabic{lstannotation}}}
\newcommand{\annotation}[1]{\refstepcounter{lstannotation}\label{#1}\thelstannotation}


\definecolor{mGreen}{rgb}{0,0.6,0}
\definecolor{mGray}{rgb}{0.5,0.5,0.5}
\definecolor{mPurple}{rgb}{0.58,0,0.82}
\definecolor{backgroundColour}{rgb}{0.95,0.95,0.92}

\lstdefinestyle{CStyle}{
    backgroundcolor=\color{backgroundColour},
    commentstyle=\color{mGreen},
    keywordstyle=\color{magenta},
    numberstyle=\tiny\color{mGray},
    stringstyle=\color{mPurple},
    basicstyle=\footnotesize\ttfamily,
    breakatwhitespace=false,
    breaklines=true,
    captionpos=b,
    keepspaces=true,
    numbers=left,
    numbersep=5pt,
    showspaces=false,
    showstringspaces=false,
    showtabs=false,
    tabsize=1,
    language=C++
  }

\let\origthelstnumber\thelstnumber
\makeatletter
\newcommand*\Suppressnumber{%
  \lst@AddToHook{OnNewLine}{%
    \let\thelstnumber\relax%
     \advance\c@lstnumber-\@ne\relax%
    }%
}

\newcommand*\Reactivatenumber[1]{%
  \setcounter{lstnumber}{\numexpr#1-1\relax}
  \lst@AddToHook{OnNewLine}{%
   \let\thelstnumber\origthelstnumber%
   \refstepcounter{lstnumber}%
  }%
}


\makeatother

\makeatletter
\newcommand{\setword}[2]{%
  \phantomsection
  #1\def\@currentlabel{\unexpanded{#1}}\label{#2}%
}
\makeatother

\newcommand{\newpar}[1]{\vspace{0.1cm}\noindent\textbf{#1.}\xspace}

%-------------------------------------------------------------------------------

%-------------------------------------------------------------------------------
\begin{document}
%-------------------------------------------------------------------------------
\onecolumn
%don't want date printed
\date{}

% make title bold and 14 pt font (Latex default is non-bold, 16 pt)
\title{\Large \bf SafeFetch: Practical Double-Fetch Protection\\with Kernel-Fetch Caching}

\section{Performance Artifact}
In the following subsections you will find results pertaining to the main performance
evaluation for \tech as follows: LMBEnch syscall overhead (see Section), 
LMBench bandwidth overheads (see Section), OSBench and Phoronix.
For each benchmark the artifact will compute the results from the paper (using our 
collected raw data).
Additionally, if the evaluator runs the artifact, additional tables/graphs showing 
the results obtained locally on the evaluator's machine are shown alongside the 
results in the paper.
For this, the evaluator must at least run the performance evaluation for SafeFetch (default)
configuration on her/his machine.
The evaluator can additionally run the whitelist and/or midas configuration to obtain the 
respective results on her/his machine.
\subsection{LMBench syscall performance results}
\begin{table}[!h]
  \caption{LMBench latency results (from the main paper).}
  \begin{center}
  \resizebox{0.5\columnwidth}{!}{%
  \input{tables/lmbench_performance_1}
  %
  }
\end{center}
\label{tab:lmbench-performance-paper}
\end{table}

\begin{table}[!h]
  \caption{LMBench latency results (ran locally on the machine).}
  \begin{center}
  \resizebox{0.5\columnwidth}{!}{%
  \begin{tabular}{l|r|rr|rr|rr}
\multirow{3}{*}{\textbf{Benchmark}} &\multicolumn{1}{c|}{\textbf{Baseline}}& \multicolumn{2}{c|}{\textbf{SafeFetch-Default}}& \multicolumn{2}{c|}{\textbf{SafeFetch-Whitelist}}& \multicolumn{2}{c|}{\textbf{Midas}}\\
&\multicolumn{1}{c|}{\textbf{($\mu$seconds)}}& \multicolumn{2}{c|}{\textbf{(\%)}}& \multicolumn{2}{c|}{\textbf{(\%)}}& \multicolumn{2}{c|}{\textbf{(\%)}}\\
&&\multicolumn{1}{c}{\textbf{ovr.}}&\multicolumn{1}{c|}{\textbf{stddev}}&\multicolumn{1}{c}{\textbf{ovr.}}&\multicolumn{1}{c|}{\textbf{stddev}}&\multicolumn{1}{c}{\textbf{ovr.}}&\multicolumn{1}{c|}{\textbf{stddev}}\\
\hline
\end{tabular}

  %
  }
\end{center}
\label{tab:lmbench-performance-local}
\end{table}

\newpage
\subsection{LMBench bandwidth results}

\begin{table}[!h]
  \caption{LMBench bandwidth results (from the main paper).}
  \begin{center}
  \resizebox{0.5\columnwidth}{!}{%
  \input{tables/lmbench_bandwidth_1}
  %
  }
\end{center}
\label{tab:lmbench-bandwidth-paper}
\end{table}

\begin{table}[!h]
  \caption{LMBench bandwidth results (ran locally on the machine).}
  \begin{center}
  \resizebox{0.5\columnwidth}{!}{%
  \input{tables/lmbench_bandwidth_0}
  %
  }
\end{center}
\label{tab:lmbench-bandwidth-local}
\end{table}

\newpage
\subsection{OSBench performance results}
\begin{figure*}[!h]
  \hfill
  \subfigure[OSBench(paper)]{\includegraphics[width=0.45\columnwidth]{figs/osbench_performance_1.pdf}}
  \hfill
  \subfigure[OSBench(local)]{\includegraphics[width=0.45\columnwidth]{figs/osbench_performance_0.pdf}}
  \hfill
  \centering
  \caption{OSBench performance from paper (a) and ran locally on the evaluator's machine (b).}
  \label{fig:phoronixosbench}
  \end{figure*}
\newpage

\subsection{Phoronix performance results}
\begin{figure*}[!h]
  \hfill
  \subfigure[Phoronix(paper)]{\includegraphics[width=0.45\columnwidth]{figs/phoronix_performance_1.pdf}}
  \hfill
  \subfigure[Phoronix(local)]{\includegraphics[width=0.45\columnwidth]{figs/phoronix_performance_0.pdf}}
  \hfill
  \centering
  \caption{Phoronix performance from paper (a) and ran locally on the evaluator's machine (b).}
  \label{fig:phoronixosbench}
  \end{figure*}

\newpage
\section{Security Artifact}
In the following subsection the evaluator can see the number of reproductions of \texttt{CVE-2016-6516} 
across 5 iterations when running the artifact on a baseline kernel (that doesn't use SafeFetch) and on a 
SafeFetch protected kernel.
Additionally we also show the average per-iteration reproductions of the bug.
The evaluator must initially run the security artifact on it's local machine as 
explained in the artifact description (without running the artifact only the table headers are generated).
On a baseline kernel the \textbf{Reproductions} column should be non-zero, signaling that the CVE was 
reproduced.
Conversely, on a SafeFetch protected kernel the \textbf{Reproductions} column should only contain zeroes
which signals that SafeFetch is able to thwart the bug.

\begin{table}[!h]
  \caption{\texttt{CVE-2016-6516} reproductions across 5 iterations. (baseline kernel)}
  \begin{center}
  \resizebox{0.25\columnwidth}{!}{%
  \input{tables/security_baseline.tex}
  %
  }
\end{center}
\label{tab:security-baseline-local}
\end{table}
\begin{table}[!h]
  \caption{\texttt{CVE-2016-6516} reproductions across 5 iterations. (kernel defended with SafeFetch)}
  \begin{center}
  \resizebox{0.25\columnwidth}{!}{%
  \input{tables/security_safefetch.tex}
  %
  }
\end{center}
\label{tab:security-safefetch-local}
\end{table}


%%%%%%%%%%%%%%%%%%%%%%%%%%%%%%%%%%%%%%%%%%%%%%%%%%%%%%%%%%%%%%%%%%%%%%%%%%%%%%%%
\end{document}
%%%%%%%%%%%%%%%%%%%%%%%%%%%%%%%%%%%%%%%%%%%%%%%%%%%%%%%%%%%%%%%%%%%%%%%%%%%%%%%%

%%  LocalWords:  endnotes includegraphics fread ptr nobj noindent
%%  LocalWords:  pdflatex acks
